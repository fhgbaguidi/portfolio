\documentclass[11pt,a4paper,sans]{moderncv}

% ----------------------------------------------------------------------------------
% CONFIGURATION DU THÈME ET DES COULEURS
% ----------------------------------------------------------------------------------
\moderncvstyle{classic} 
\moderncvcolor{blue} % Initialisation obligatoire

% OVERRIDE COULEUR (INDIGO PORTFOLIO #4F46E5)
\usepackage{xcolor}
\definecolor{color1}{HTML}{4F46E5} 

% Encodage et mise en page
\usepackage[utf8]{inputenc}
\usepackage[scale=0.75]{geometry}
\usepackage[french]{babel}

% ----------------------------------------------------------------------------------
% DONNÉES PERSONNELLES
% ----------------------------------------------------------------------------------
\name{Fréjus}{GBAGUIDI}
\title{Senior Java Backend Engineer \& Cloud Developer}
\address{Lille, France (Ouvert au Remote)}{}{Freelance}
\phone[mobile]{+33 6 50 27 94 29}
\email{fh.gbaguidi@gmail.com}
\social[linkedin]{fgbaguidi}
\social[github]{fhgbaguidi}
\extrainfo{Portfolio: \href{https://fhgbaguidi.github.io/portfolio/}{fhgbaguidi.github.io/portfolio}}

% ----------------------------------------------------------------------------------
% CONTENU
% ----------------------------------------------------------------------------------
\begin{document}

\makecvtitle

% Profil
\section{Profil}
\cvitem{}{Ingénieur logiciel senior avec plus de \textbf{10 ans d'expérience}, je suis spécialisé dans la conception d'architectures distribuées résilientes. Expert de l'écosystème \textbf{Java (25, Spring Boot 3+)}, j'accompagne la modernisation des SI critiques. Je conçois des systèmes performants, scalables et maintenables en appliquant les principes du Software Craftsmanship et du GitOps.}

% Compétences Techniques (Strictement aligné sur Portfolio HTML)
\section{Compétences Techniques (Stack 2026)}
\cvitem{\textbf{Backend}}{\textbf{Java 25}, Spring Boot 3+, JPA/Hibernate}
\cvitem{\textbf{Architecture}}{\textbf{CQRS}, \textbf{CDC (Change Data Capture)}, Onion/Clean Architecture, Microservices, DDD}
\cvitem{\textbf{Data \& Msg}}{\textbf{Kafka}, \textbf{Debezium}, PostgreSQL, Elasticsearch}
\cvitem{\textbf{Cloud \& Ops}}{\textbf{Kubernetes}, \textbf{GitOps (ArgoCD)}, Docker, GitHub Actions, Terraform, GCP}
\cvitem{\textbf{Methodology}}{TDD / BDD, Software Craftsmanship, Code Review, \textbf{Jira}}

% Expériences
\section{Expériences Professionnelles}

\cventry{Nov 2022 -- Présent}{Senior Java Engineer}{Decathlon United (International)}{Lille / Hybride}{}{
\textbf{Contexte :} Centralisation du catalogue produit pour l'ensemble des pays (United). Outil central alimentant l'Ecommerce et les Magasins.
\begin{itemize}
    \item \textbf{Stratégie Data :} Centralisation, uniformisation et structuration du catalogue produit mondial. \textbf{Enrichissement et mise à disposition de plusieurs centaines de millions de fiches produits aux différents canaux.} Garantie de l'intégrité automatisée via écoute de référentiels externes.
    \item \textbf{Modernisation :} Réécriture complète du legacy Haskell vers \textbf{Java 25} \& Spring Boot 3+.
    \item \textbf{Architecture Event-Driven :} Implémentation de \textbf{CDC (Change Data Capture)} avec Debezium sur PostgreSQL pour alimenter Kafka en temps réel.
    \item \textbf{Resilience :} Mise en place du pattern \textbf{Transactional Outbox} pour garantir la cohérence des données (CQRS). Le Query Service consomme une base de projection dédiée.
    \item \textbf{Ops :} Pipelines CI/CD modernes et déploiement \textbf{GitOps} sur Kubernetes. Suivi via \textbf{Jira}.
\end{itemize}}

\cventry{Mai 2019 -- Oct 2022}{Senior Software Engineer}{ADEO Services}{Lille}{}{
\textbf{Contexte :} Écosystème backend pour Leroy Merlin et le groupe ADEO.
\begin{itemize}
    \item Construction de microservices Java robustes dans un environnement Cloud (GCP).
    \item Contribution à la stratégie DevOps : automatisation des tests et des déploiements.
    \item Application des bonnes pratiques de Software Craftsmanship et gestion via \textbf{Jira}.
\end{itemize}}

\cventry{Sept 2018 -- Avr 2019}{Tech Lead / Lead Developer}{Norauto International (BeAPI)}{Lille}{}{
\begin{itemize}
    \item Pilotage technique de l'équipe transverse (API Factory).
    \item Définition des standards de code et évangélisation du Clean Code.
    \item Industrialisation des processus de livraison (CI/CD) pour réduire le Time-to-Market.
\end{itemize}}

\cventry{Jan 2018 -- Sept 2018}{Backend Engineer (Golang)}{Norauto International (BeAPI)}{Lille}{}{
\begin{itemize}
    \item Conception et développement de microservices critiques en \textbf{Golang}.
    \item Sécurisation des échanges API via protocoles OAuth2 / OpenID Connect.
    \item Tuning des performances (Memory management, Concurrency).
\end{itemize}}

\cventry{Août 2017 -- Déc 2017}{Architecte Technique Ecommerce}{Norauto International}{Lille}{}{
\begin{itemize}
    \item Analyse de la dette technique et élaboration de la roadmap de refactoring.
    \item Amélioration de la Developer Experience (DX) via outillage interne.
\end{itemize}}

\cventry{Jan 2017 -- Juil 2017}{Lead Developer \& DevOps}{Norauto International}{Lille}{}{
\begin{itemize}
    \item Design et implémentation de la stack \textbf{ELK} (Elasticsearch, Logstash, Kibana).
    \item Gestion des incidents et supervision de la production (Run management).
\end{itemize}}

\cventry{Sept 2015 -- Déc 2016}{Software Engineer}{Norauto International}{Lille}{}{
\begin{itemize}
    \item Développement de modules backend Java/J2EE.
    \item Création d'outils de monitoring marketing temps réel.
\end{itemize}}

% Formation
\section{Formation}
\cventry{2013 -- 2015}{Master TIIR (Technologies de l'Information et Ingénierie du Web et Réseau)}{Université de Lille 1}{Lille}{}{}
\cventry{2012 -- 2013}{Licence Informatique}{Université de Lille 1}{Lille}{}{}
\cventry{2008 -- 2011}{Licence Informatique (MIAGE) \& DUT GTR}{UPIB}{Cotonou}{}{}

% Langues & Intérêts
\section{Langues \& Intérêts}
\cvitem{Langues}{Français (Natif), Anglais (Technique/Professionnel), Fon}
\cvitem{Intérêts}{Veille Technologique, Musique, Cuisine, Voyages}

\end{document}
